\documentclass[a4paper, 11pt, oneside]{article}

\usepackage[utf8]{inputenc}
\usepackage[T1]{fontenc}
\usepackage[french]{babel}
\usepackage{array}
\usepackage{shortvrb}
\usepackage{listings}
\usepackage[fleqn]{amsmath}
\usepackage{amsfonts}
\usepackage{fullpage}
\usepackage{enumerate}
\usepackage{graphicx}             % import, scale, and rotate graphics
\usepackage{subfigure}            % group figures
\usepackage{alltt}
\usepackage{url}
\usepackage{indentfirst}
\usepackage{eurosym}
\usepackage{listings}
\usepackage{color}
\usepackage[table,xcdraw,dvipsnames]{xcolor}

% Change le nom par défaut des listing
\renewcommand{\lstlistingname}{Extrait de Code}

% Change la police des titres pour convenir à votre seul lecteur
\usepackage{sectsty}
\allsectionsfont{\sffamily\mdseries\upshape} 
% Idem pour la table des matière.
\usepackage[nottoc,notlof,notlot]{tocbibind} 
\usepackage[titles,subfigure]{tocloft} 
\renewcommand{\cftsecfont}{\rmfamily\mdseries\upshape}
\renewcommand{\cftsecpagefont}{\rmfamily\mdseries\upshape} 

\definecolor{mygray}{rgb}{0.5,0.5,0.5}
\newcommand{\coms}[1]{\textcolor{MidnightBlue}{#1}}

\lstset{
    language=C, % Utilisation du langage C
    commentstyle={\color{MidnightBlue}}, % Couleur des commentaires
    frame=single, % Entoure le code d'un joli cadre
    rulecolor=\color{black}, % Couleur de la ligne qui forme le cadre
    stringstyle=\color{RawSienna}, % Couleur des chaines de caractères
    numbers=left, % Ajoute une numérotation des lignes à gauche
    numbersep=5pt, % Distance entre les numérots de lignes et le code
    numberstyle=\tiny\color{mygray}, % Couleur des numéros de lignes
    basicstyle=\tt\footnotesize, 
    tabsize=3, % Largeur des tabulations par défaut
    keywordstyle=\tt\bf\footnotesize\color{Sepia}, % Style des mots-clés
    extendedchars=true, 
    captionpos=b, % sets the caption-position to bottom
    texcl=true, % Commentaires sur une ligne interprétés en Latex
    showstringspaces=false, % Ne montre pas les espace dans les chaines de caractères
    escapeinside={(>}{<)}, % Permet de mettre du latex entre des <( et )>.
    inputencoding=utf8,
    literate=
  {á}{{\'a}}1 {é}{{\'e}}1 {í}{{\'i}}1 {ó}{{\'o}}1 {ú}{{\'u}}1
  {Á}{{\'A}}1 {É}{{\'E}}1 {Í}{{\'I}}1 {Ó}{{\'O}}1 {Ú}{{\'U}}1
  {à}{{\`a}}1 {è}{{\`e}}1 {ì}{{\`i}}1 {ò}{{\`o}}1 {ù}{{\`u}}1
  {À}{{\`A}}1 {È}{{\`E}}1 {Ì}{{\`I}}1 {Ò}{{\`O}}1 {Ù}{{\`U}}1
  {ä}{{\"a}}1 {ë}{{\"e}}1 {ï}{{\"i}}1 {ö}{{\"o}}1 {ü}{{\"u}}1
  {Ä}{{\"A}}1 {Ë}{{\"E}}1 {Ï}{{\"I}}1 {Ö}{{\"O}}1 {Ü}{{\"U}}1
  {â}{{\^a}}1 {ê}{{\^e}}1 {î}{{\^i}}1 {ô}{{\^o}}1 {û}{{\^u}}1
  {Â}{{\^A}}1 {Ê}{{\^E}}1 {Î}{{\^I}}1 {Ô}{{\^O}}1 {Û}{{\^U}}1
  {œ}{{\oe}}1 {Œ}{{\OE}}1 {æ}{{\ae}}1 {Æ}{{\AE}}1 {ß}{{\ss}}1
  {ű}{{\H{u}}}1 {Ű}{{\H{U}}}1 {ő}{{\H{o}}}1 {Ő}{{\H{O}}}1
  {ç}{{\c c}}1 {Ç}{{\c C}}1 {ø}{{\o}}1 {å}{{\r a}}1 {Å}{{\r A}}1
  {€}{{\euro}}1 {£}{{\pounds}}1 {«}{{\guillemotleft}}1
  {»}{{\guillemotright}}1 {ñ}{{\~n}}1 {Ñ}{{\~N}}1 {¿}{{?`}}1
}
\newcommand{\tablemat}{~}

%%%%%%%%%%%%%%%%% TITRE %%%%%%%%%%%%%%%%
% Complétez et décommentez les définitions de macros suivantes :
 \newcommand{\intitule}{Premier milestone}
 \newcommand{\GrNbr}{6}
 \newcommand{\PrenomUN}{Antoine}
 \newcommand{\NomUN}{Demany}
 \newcommand{\PrenomDEUX}{Guillaume}
 \newcommand{\NomDEUX}{Delacollette}
% Décommentez ceci si vous voulez une table des matières :
% \renewcommand{\tablemat}{\tableofcontents}

%%%%%%%% ZONE PROTÉGÉE : MODIFIEZ UNE DES DIX PROCHAINES %%%%%%%%
%%%%%%%%            LIGNES POUR PERDRE 2 PTS.            %%%%%%%%
\title{INFO0947: \intitule}
\author{Groupe \GrNbr : \PrenomUN~\textsc{\NomUN}, \PrenomDEUX~\textsc{\NomDEUX}}
\date{}
\begin{document}
\maketitle
\newpage
\tablemat
\newpage
%%%%%%%%%%%%%%%%%%%% FIN DE LA ZONE PROTÉGÉE %%%%%%%%%%%%%%%%%%%%

%%%%%%%%%%%%%%%% RAPPORT %%%%%%%%%%%%%%%
% Écrivez votre rapport ci-dessous.

\section{\LARGE \bfseries Définition du problème : }
	\subsection{Trouver un même préfixe et suffixe d'un tableau d'entier}
		\begin{itemize}
			\item Données en entrées : un tableau T à N valeurs entières
			\item Résultats attendus : 
			\begin{itemize}
				\item[$\star$] Soit 0 si il n'y a pas de préfixe/suffixe au tableau
				\item[$\star$] Soit la taille du plus grand préfixe/suffixe du tableau en entrée
			\end{itemize}
		\end{itemize}
		
\section{\LARGE \bfseries Analyse du problème : }
	\subsection{Découpe en sous-problèmes}
		\begin{itemize}
		
			\item SP0 : Vérifier que les données entrées sont correctes (assert)
			\item SP1 : Initialiser nos variables (dont "result" qui sera la valeur à retourner)
			\item SP2 : Boucle principale qui balaye le tableau
				\newline
				Qui va balayer tant que i $\leq$ N-1
				\begin{figure}[!h]
					\centering
					\begin{tabular}{l|llr|ll|l}
						& 0 &  & \multicolumn{1}{r|}{i-1} & i &  & N-1 \\ \cline{2-6}
						boucle: & \cellcolor[HTML]{FFCC67} & \cellcolor[HTML]{FFCC67} & \cellcolor[HTML]{FFCC67} &  &  &  \\ \cline{2-6}
					\end{tabular}
					\caption{SP2}
					\label{fig:exemple}
				\end{figure}
				\begin{itemize}
					\item SP2a : recupepe le préfixe (de i+1 de long, donc 1 chiffres, puis 2, etc) et le comparer au suffixe
					\item SP2b : si le préfixe égale le suffixe, stocker la valeur de i dans une variable (=> result)
                    \item SP2c : truc a la con
				\end{itemize}
			\item SP3 : Retourner la valeur de "result" (si aucun préfixe/suffixe n'a été trouvé, result est égale à sa valeur d'initialisation: 0)
		\end{itemize}
	\subsection{Shéma des sous-problèmes}
		SP0 $\rightarrow$ SP1 $\rightarrow$ ((SP2a $\rightarrow$ SP2b) $\subset$ SP2) $\rightarrow$ SP3

\section{\LARGE \bfseries Écriture du code :}
	[...]

\section{\LARGE \bfseries Tests : }
	[...]


\end{document}














